\documentclass[a4paper]{article}

\usepackage[T2A]{fontenc}
\usepackage{cmap}
\usepackage[utf8]{inputenc}
\usepackage[english, russian]{babel}
\usepackage[a-1b]{pdfx}
\usepackage[warn]{mathtext}
\usepackage{amsmath}
\usepackage{amssymb}
\usepackage{latexsym}
\usepackage{amssymb,amsmath}
\title{Number theory}
\author{M-18 }
\date{June 2020}

\usepackage{natbib}
\usepackage{graphicx}
\hyphenation{словобезпереносов}
\begin{document}

\maketitle

\section{Алгоритм Евклида}
\textbf{Определение 1.1}
(Целочисленное деление с остатком).

Пусть \textit{a} \in  \mathbb{Z},  b \in  \mathbb{N}  \text{. Тогда целочисленное деление с остатком вида вычисляется по формуле}

\begin{center}
  \textit{a=bq+r},
\end{center}

где \textit{q} \in \mathbb{Z} - \text{частное}, r \in \mathbb{N}\cup\lbrace0\rbrace, 0\le r<b - \text{остаток}

\noindent\textbf{Утверждение 1.1} Всякое целое \textit{a} представляется единственным \mbox{образом} через положительное целое число \textit{b} в форме

\[
\textit{a=bq+r} , 0 \le r<b.
\]
\textit{Доказательство.} Действительно, одно представление числа а в такой форме получим, взяв \textit{bq} равным наибольшему кратному \textit{b} числу не превосходящему \textit{a}.

\[
q = \lbrack\frac{a}{b}\rbrack.
\]

\textit{q}- \text{целая часть снизу, т.е. нибольшее число не превосходящее данное. Откуда}

\[
\textit{r=a--bq}
\]

Допустим, что разложений два 

\[
a=bq_{1}+r_{1}, 0 \le r_{1} <b
\]

\[
a=bq_{2}+r_{2}, 0 \le r_{2} <b
\]

получим

\[
0=b(q_{1}-q_{2})+r_{1}-r_{2}
\]

отсуда следует, что r_{1}-r_{2} \text{ кратно } \textit{b}.\text{ Но ввиду того, что } |r_{1}-r_{2}|< \textit{b},\\ \text{последнеее возможно лишь при } r_{1}-r_{2}=0 => r_{1}=r_{2}.\\\text{Откуда вытекает также } q_{1}=q_{2}.\\ \text{Соответственно, любое целое число можно \mbox{разделить} на натуральное.}

\noindent\textbf{Определение 1.2}
(Алгоритм Евклида. Нахождение НОД)

Пусть a,b \in  \mathbb{Z}. \text{Тогда по Определению 1.1 находим ряд равенств:}

\[
\left\{
\begin{aligned}
a=bq_{1}+r_{2}, 0 \le r_{2} <b \\
b=bq_{2}+r_{3}, 0 \le r_{1} <b \\
. \\
. \\
. \\
r_{n-2}=r_{n-1}q_{n-1}+r_{n}, r_{n-1}<r_{n}\\
r_{n-1}=r_{n}q_{n}+r_{n+1}\\                           
\end{aligned}
\right.
\]

\begin{flushright}
  (1)
\end{flushright}

Алгоритм завершается при r_{n+1}

\noindent\textbf{Утвержделение 1.2} В алгоритме Евклида r_{n}=\text{НОД(a,b). Далее введем}\\ \text{ обозначение НОД(a,b) через (a,b)}

\textit{Доказательство.} 1. Докажем, что r_{n} - \text{делитель числа а. Рассмотрим получиашийся}\\ \text{ ряд (1) снизу вверх:}

\[
r_{n}|r_{n-1}, r_{n-1}|r_{n-2} => r_{n}|r_{n-1} ... =>  r_{n}|a, r_{n}|b
\]

2. Докажем, что r_{n} - \text{наибольший общий делитель. Рассмотрим ряд (1) сверху вниз:}

Пусть делитель d>r_{n}.\text{Тогда}

\[
d|r_{2}, d|r_{3}, ... , d|r_{n}
\]

Противоречие!

\noindent\textbf{Утвержделение 1.3} НОД делится на любой другой общий делитель. Т.е.
\[
\forall d:\lbrace d|a\wedge d|b\rbrace => d|(a,b)
\]

\textit{Доказательство.} По алгоритму Евклида:
\[
d|a, d|b => d|r_{n},
\]

где, r_{n} -\text{НОД. Что и требовалось доказать}.
\end{document}