\documentclass{article}
 
\usepackage[T2A]{fontenc}
\usepackage[utf8]{inputenc}
\usepackage[russian]{babel}
\usepackage{amsmath}
\usepackage{amsthm}

\newtheorem{theorem}{Теорема}

\title{Теоремы Эйлера и Ферма}
\author{}
\date{}

\begin{document}
\maketitle

\section{Теорема Эйлера}

\begin{theorem}[Т. Эйлера]
Если два целых положительных числа $\textbf{n}$ и $\textbf{a}$ взаимно просты, то верно следующее утверждение\\
\null\qquad\qquad\qquad $a^{\varphi(n)} \equiv 1 \pmod{n}$, \qquad где $\varphi(n)$ — функция Эйлера
\end{theorem}
 
\begin{proof}
Пусть $x_1, x_2, \ldots, x_{\varphi(n)}$ — приведенная система вычетов по модулю $n$. \\
Тогда $ax_1, ax_2, \ldots, ax_{\varphi(n)}$ — тоже приведенная система вычетов по тому же модулю $n$.
Тогда получается, что верно сравнение 
\[ x_1 \cdot x_2 \cdot \ldots \cdot x_{\varphi(n)} \equiv ax_1 \cdot ax_2 \cdot \ldots \cdot ax_{\varphi(n)} \pmod{n} \]
Очевидно, что $a$ перемножилось $\varphi(n)$ раз, поэтому имеем:
\[ x_1 \cdot x_2 \cdot \ldots \cdot x_{\varphi(n)} \equiv a^{\varphi(n)} \cdot  x_1 \cdot x_2 \cdot \ldots \cdot x_{\varphi(n)} \pmod{n} \]
Заметим, $x_i$ взаимно просты с $n$ {так как $x_i$ входит в приведенную систему вычетов по модулю $n$} и воспользуемся свойством сравнения по модулю. Получаем:
\[1  \equiv  a^{\varphi(n)} \pmod{n} \]
Что нам и требовалось доказать.
\end{proof}

\end{document}